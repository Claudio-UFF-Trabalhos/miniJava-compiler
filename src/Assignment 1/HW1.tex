\documentclass[a4paper,12pt,titlepage]{article}
\usepackage{fullpage}
\usepackage{listings}
\DeclareFontShape{OT1}{cmtt}{bx}{n}{<5><6><7><8><9><10><10.95><12><14.4><17.28><20.74><24.88>cmttb10}{}

\title{COMP 520: Compilers\\{\bf Problem set 1}}
\author{\bf Shrey Banga\\banga@cs.unc.edu}
\date{\today}

\begin{document}

\lstset{
	tabsize=2,
	basicstyle=\ttfamily,
	keywordstyle=\bfseries
}

\maketitle

\begin{enumerate}

% Question 1
\item	\texttt{Object o = "abc";}
	\begin{enumerate} 

	\item \texttt{char c = o.charAt(2);}\\
	This is not a legal Java statement, as the method \texttt{charAt} is not defined for the \texttt{Object} class to which \texttt{o} belongs.

	\item \texttt{boolean b = o.equals("abc");}\\
	This is a legal statement as the \texttt{equals} method is defined for the \texttt{Object} class. It returns \texttt{true}, which gets assigned to \texttt{b}.

	\item \texttt{String s = o;}\\
	This is not a legal Java statement as a variable of type \texttt{String} cannot hold the reference to an instance of the \texttt{Object} class, since \texttt{String} is a subclass of \texttt{Object}. It would have to be explicitly typecast into a \texttt{String} object.

	\item \texttt{char d = ((String) o).charAt(3);}\\
	This is a legal Java statement, as the object is typecast explicitly into a \texttt{String} object before calling the \texttt{charAt} method on it. However, the index passed to the method is out of the range of the string, causing it to throw a run time exception.

	\end{enumerate}

% Question 2
\item The following code shows how to access each of the four occurences of \texttt{x}:\\
	\lstinputlisting[language=Java]{Test.java}

% Question 3
\item
	\lstinputlisting[language={[x86masm]Assembler}]{gcd.asm}

\end{enumerate}

\end{document}